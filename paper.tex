\section*{Summary of the Proposal}
This research offers a joint consideration to two distinct problems in second-price auctions with common values: seller's optimal supply of information precision and bidders' incentive to acquire information. Normality assumption is adopted to allow for the utilization of different notions of stochastic ordering of signals simultaneously in a highly tractable learning environment. The main contribution is aimed to be enabling cooperative behavior among bidders through introduction of an \textit{information pool}. Resulting environment can potentially lead to novel strategic considerations between the seller and the bidders who aim to capture larger shares of the total surplus.
 

\section*{Background}
Following \cite{milgrom1982theory}'s seminal results, it's widely accepted that publicly revealing information cannot hurt the seller in the common value setting. In their setup, valuations are characterized as functions of public and/or private signals; whereas for the setting considered in this research, signals are solely information sources from which valuation can be inferred. This divergence from the mainstream literature poses considerable challenges and it's of interest of this research to investigate the robustness of classical results under this alternative characterization.

\cite{ganuza2010signal} consider optimal supply of signal precision, where for signals that can be ordered by \textit{supermodular precision}, authors show that in SPA with independent and private (but uncertain) values, optimal level of precision supplied by the auctioneer is lower than the socially efficient level. Intuitively, this is a result of a trade-off concerning seller's incentive to supply information: higher precision results in higher efficiency and increases total surplus, while simultaneously making bidders more heterogeneous and increasing winning bidder's information rents, hence lowering seller's share of surplus. Their resulting conclusion is that under private values, seller can not be relied on to disclose all socially desirable information. From the bidders' perspective, \cite{persico2000information} discusses bidders' incentives to acquire more precise information, where in a general value SPA framework it's shown that for exogenously supplied, \textit{affiliated} and \textit{accuracy ordered} signals, expected bidder payoff is non-decreasing in the accuracy of the signal. His setup also considers value as a function of signals, which is not the case here.

\section*{Goal and Objectives}

The aim of this research is to jointly consider these two frameworks through endogenizing bidders' information acquisition process by letting bidders pool their private signals that are supplied by the seller. The setting can be seen as a primitive abstraction of communication through formation of social groups in a competitive environment. While the exact scenario treated here may be hard to come by in real cases; I think it's sensible to think that personal judgments of value (especially the accuracy of the judgment) are usually shaped through direct or indirect communication. Main limitation of this research is the restriction of the setting to common values to enable bidders to learn from each others' signals, as under Bayesian updating each signal has to be conditioned on the same random variable.

Bidders' incentive to participate in the information pool is characterized as acquiring more precise information regarding the common value. Following \cite{persico2000information}'s result, participation decision might seem to be straight-forward, although with one caveat: here, a bidder's decision to obtain more precise information simultaneously allows other bidders to do the same. Thus, in particular, the effect on the distribution of second-highest bid should be of concern when analyzing the optimal participation decision. 

Regarding the seller's decision, the framework proposed in \cite{ganuza2010signal} considers private values. Since in their paper, the adverse effect of higher precision on seller's revenue is due to an increase in information rents (following higher heterogeneity of bidder valuations), it's unclear whether this effect is sustained under common values. Another concern is that, while generally increasing participation is seen as beneficial for the seller, here, it may also increase the size of the information pool, potentially leading to undesirably high precision levels for the seller. One strategic consideration here might be that seller may anticipate future communication among bidders and under-supply precision accordingly; while another one might be to restrict bidder entry.

\section*{Model}
\subsection*{Setup}
Formally, let there be $n \geq 2$ ex-ante identical risk neutral bidders who have an unknown common value $v$ for a single indivisible object that's being sold in a second-price sealed-bid auction with no reserve price. Bidders treat the common value as a random variable $V$ and hold the common prior belief such that $V \sim N(v,\rho_{0}^{-1})$. What's explicitly known to bidders (and what is of central importance) is the precision level of this belief. Before the auction, seller has the opportunity to produce private signals $(X_{i}^\rho)_{i=1}^n$ of precision $\rho$ that are identically and independently distributed with  $X_i \mid V \sim N(v,\rho^{-1})$. Upon receiving their signals (in addition to learning from their own signal) bidders may choose to participate in an information pool, where they can truthfully share their private signal realizations and learn from other bidders' signal realizations in the pool. The learning process is characterized by Bayesian updating and hence if first $n \geq k \geq 2$ bidders choose to participate in the information pool, each bidder in the pool then have $k-1$ additional signals conditioned on $V$ that they can use to sequentially update their prior beliefs. Bidder $i$'s posterior belief on the common value if she chooses not to participate in the information pool is
\[V \mid X_i=x_i \sim N(v,[\rho_0 + \rho]^{-1})\] 
whereas, if she chooses to participate with $k-1$ other bidders is
\[V \mid X_1=x_1,...,X_k=x_k \sim N(v,[\rho_0 + k\rho]^{-1})\] 
For ease of computation, it's without loss to think about updating the prior belief with $k$ signals from distribution $N(v, \rho^{-1})$, as updating it with one \textit{aggregate} signal from distribution $N(v, [k\rho]^{-1})$ in terms of the posterior distributions they generate. Let $\Phi$, $\Phi^I$, $\Phi'^I$, $\Phi_k^P$ and $ \Phi_k'^P$ denote the cumulative distribution functions of: prior belief; individual signal; prior belief updated with individual signal realization; aggregate signal from a pool of size $k$ and prior belief updated with this aggregate signal realization, respectively. Then, pool participation decision can be partially reformulated as acquiring the signal with conditional c.d.f. $\Phi_k^P$ or $\Phi^I$, where latter is a \textit{mean-preserving spread} of the former.

\subsection*{Bidders' Incentives}
Bidders know about the precision level chosen by the seller and make the pool participation decision based on the anticipated precision level of their posterior beliefs. Participating in the pool allows access to additional signal realizations, hence, enables bidders to construct a more precise posterior belief. This opportunity simultaneously allows other bidders in the pool to do the same, which might potentially be a drawback.

As a first attempt, it's instructive to check whether full participation in the information pool can be an equilibrium. Assuming all bidders but bidder $i$ participate in the information pool, if bidder $i$ chooses to participate, each bidder would then symmetrically update their prior belief using $n$ signal realizations. Then, bidder $i$'s optimal bid in the SPA given signal realizations $\boldsymbol{x}=(x_1,...,x_n)$ and her resulting posterior belief can be characterized as,
\[b_i^P(\boldsymbol{x}) \in \arg\max_{b} \int_{-\infty}^{\infty} [V - W(\boldsymbol{x})]\mathbf{1}_{\{W(\boldsymbol{x})<b\}} \,d\Phi_n'^P(V\mid \boldsymbol{X}=\boldsymbol{x})\]
where $W(\boldsymbol{x})=max_{i \neq j}b_j(\boldsymbol{x})$ and $\mathbf{1}$ is the indicator function. This characterization allows the bidder to adjust her bid to win as long as her expected valuation is larger than the price paid. In the IPV setting, knowledge of the true valuation allows the bidder to perfectly adjust this level of optimal bid, which is equal to the value itself. As this is not the case here, expectation is considered instead, which makes the precision of belief crucial. With a more precise judgment on the true value, bidders should be able to increase their bid for higher winning probability with a lower risk of bidding above their value.

If she chooses not to participate, she will only update her prior with her individual signal, where other bidders will use $n-1$ signals to update their priors before placing their bids. Hence, compared to the first scenario, accuracy will be lower for all bidders, but (potentially) much lower for bidder $i$. In this case, bidder $i$'s optimal bid is
\[b_i^I(x_i) \in \arg\max_{b} \int_{-\infty}^{\infty} [V - W(\boldsymbol{x_{-i}})]\mathbf{1}_{\{W(\boldsymbol{x_{-i}})<b\}} \,d\Phi'^I(V\mid X_i=x_i)\]
where $W(\boldsymbol{x_{-i}})=max_{i \neq j}b_j(\boldsymbol{x_{-i}})$ and $\boldsymbol{x_{-i}}=(x_1,...,x_{i-1},x_{i+1},...,x_n)$. It should be noted that signal realizations affect bids primarily through posterior beliefs that they shape. In particular and in contrast to the mainstream literature, I expect equilibrium bid functions to be indifferent to specific signal realizations and that they only change according to the number of accessible signals (i.e. according to the precision of posterior belief). 

Ultimately, it's desired to compare resulting payoffs in these two situations to determine whether it's beneficial for bidder $i$ to participate in the pool. One approach to evaluate different payoffs can be through reformulating the problem into the general decision problem setting given in \cite{persico2000information}, although it's unclear whether their method holds with an arbitrary number of bidders. Alternatively, it seems sensible to approach the solution of stochastic maximization problems by potentially imposing additional structure on the (symmetric) bidding functions to compare resulting payoffs in two cases. Due to the tractable environment, distribution of second-order statistic can also be explicitly calculated in each case to determine an expectation for the price level.

\subsection*{Seller's Incentive}
Seller chooses the level of precision $\rho$ as to maximize her expected revenue, which is equivalent to maximizing the second-highest bid. Before approaching this problem explicitly, it's essential to construct an equilibrium bidding behavior involving bid functions and participation decision for an arbitrary level of precision. If such an equilibrium can be constructed, then the optimum level of precision may be computed from the distribution of second-highest bid.

Due to the common value setting, it might seem that supplying the highest precision should be optimal for the seller. Then, posterior beliefs would be single-point distributions, where each bidder exactly knows the true value and bids away their information rents resulting in full revenue extraction by the seller. This general conclusion doesn't apply to this setting for one simple reason: bidders may bid above the common value in the presence of uncertainty. For simplicity, assume each bidder participates in the pool and bids randomly according to the resulting posterior belief on the value. Then, expectation of the second-highest bid equals to the expectation of second-order statistic of $n$ draws from $\Phi_n'^P$. This statistic then has the following cumulative distribution function\footnote{See \cite{krishna2010auction}}
\[F_2(y)=n[\Phi_n'^P(y)]^{n-1}-(n-1)[\Phi_n'^P(y)]^n\]
which can be used to explicitly compute the expected revenue. It should be clear without calculation that expected price can be greater than the mean and hence greater than the common value.



\subsection*{Alternative Characterization}
Main challenge of this research is due to the common value setting, where desirable properties of SPA aren't easily accessible. As mentioned, this constraint follows from the restriction of Bayesian updating. Alternatively, it may be possible to utilize other methods for belief updating that may potentially allow for a private value setup. Additionally, properties of supermodular precision, accuracy order, \textit{integral order} and second-order stochastic dominance can be used to obtain results, as mean-preserving spreads of normal distributions satisfy these stochastic orders.

